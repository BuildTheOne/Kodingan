\documentclass[a4paper,12pt]{article}
\usepackage[utf8]{inputenc} %codification of the document
\usepackage{mathtools}
\usepackage{amsfonts}

\begin{document}

\title{System of Linear Equations}
% \maketitle

\section*{Introduction to System of Linear Equations}
\subsection*{Linear Equation}
\begin{itemize}
  \item \textbf{Linear equation} with \(n\) \textbf{unknown} \(x_1, x_2,...,x_n\) is an equation of the form \(a_1x_1 + a_2x_2 + ... + a_nx_n = b\)  with \(a_1, a_2,\dots, a_n\) are \textbf{constant} (\(\mathbb{R}\)) \\ For example :
        \begin{itemize}
          \(x_1+3x_2+4x_2=5\)
        \end{itemize}
  \item Components of Linear Equation
        \begin{itemize}
          \item Unknown
          \item Constant
          \item Equal Sign (\(=\))
        \end{itemize}
\end{itemize}

\subsection*{System of Linear Equation}
\begin{itemize}
  \item \textbf{System of Linear Equation} or \textbf{Linear System} is a \textbf{finite set} of \textbf{linear equations} that involves the \textbf{same unknowns}
  \item Components of Linear System
        \begin{itemize}
          \item Finite Set
          \item Linear Equations
          \item Same Unknowns
        \end{itemize}
\end{itemize}

\subsection*{Solution}
\begin{itemize}
  \item Linear System is a \underline{representation of a problem}, so it must have a solution
  \item The solution in the system of linear equation is a set of numbers that is substituted to the unknowns, then the system is satisfied. Or in another word the solution satisfied all the linear equations in the system
  \item Types of linear system by its solution
        \begin{enumerate}
          \item Has exact one solution / the lines intersect at exactly one point
          \item Has infinitely many solution / the lines parallel
          \item Has no solution / inconsistent / has no intersect
        \end{enumerate}
\end{itemize}

\section*{Forms of Linear System}
\begin{itemize}
  \item General
        \begin{center}
          $$a_{11}x_1+a_{12}x_2+a_{13}x_3+\dots+a_{1n}x_n=b_1$$
          $$a_{21}x_1+a_{22}x_2+a_{23}x_3+\dots+a_{2n}x_n=b_2$$
          \vdots
          $$a_{m1}x_1+a_{m2}x_2+a_{m3}x_3+\dots+a_{mn}x_n=b_m$$
        \end{center}
  \item Matrix Equation: \(Ax=b\)
        \begin{center}
          \(\begin{bmatrix}
            a_{11} & a_{12} & a_{13} & \dots & a_{1n} \\
            a_{11} & a_{12} & a_{13} & \dots & a_{1n} \\
            \vdots                                    \\
            a_{11} & a_{12} & a_{13} & \dots & a_{1n}
          \end{bmatrix} \begin{bmatrix}
            x_1 \\ x_2 \\ \dots \\ x_n
          \end{bmatrix} = \begin{bmatrix}
            b_1 \\ b_2 \\ \dots \\ b_n
          \end{bmatrix}\)
        \end{center}
  \item Augmented Matrix: \([A|b]\)
        \begin{center}
          \(\left[\begin{array}{ccccc|c}
              a_{11} & a_{12} & a_{13} & \dots & a_{1n} & b_1    \\
              a_{21} & a_{22} & a_{23} & \dots & a_{2n} & b_2    \\
              \vdots &        &        &       &        & \vdots \\
              a_{m1} & a_{m2} & a_{m3} & \dots & a_{mn} & b_m    \\
            \end{array}\right] \)
        \end{center}
        If \(m = \text{number of linear equations}\) and \(n = \text{number of unknowns}\), then augmented matrix would have the order \(m\times (n+1)\) and coefficient matrix have the order \(m\times n\)
\end{itemize}

\section*{Method of Linear System Solution}
\begin{itemize}
  \item There are methods to find solution of a linear system (if any).
        \begin{enumerate}
          \item Elimination-Substitution
          \item Geometric
          \item Gauss-Jordan Elimination
        \end{enumerate}
        (Elimination-substitution and geometric methods already discussed in high school, so it will not be discussed too much)
  \item The disadvantage of elimination-substitution and geometric: A big linear system
  \item[] For example, consider this linear system :
    \begin{center}
      \(a+2b+3c+5d+8e=-7\) \\
      \(2a-3b-5c+4d+3e=9\) \\
      \(-6a-8b+c+2d+7e=-22\) \\
      \(-3a+5b+c-9d+8e=4\) \\
      \(7a-4b+3c+8d-e=12\)
    \end{center}
  \item[] This linear system has more than 3 unknowns and 3 equations, so elimination-substition method are not effective and geometric method is hard to use.
  \item Solution: Gauss-Jordan Elimination
\end{itemize}

\section*{Gauss-Jordan Elimination}
\subsection*{Elementary Row Operation}
\subsubsection*{Linear System Equivalence}
Two linear systems are equivalence if it has the same solution
\subsubsection*{Elementary Row Operation}
\begin{itemize}
  \item There are 3 operations that don't change the linear system solution
        \begin{enumerate}
          \item Multiply an equation through by a nonzero constant
          \item Interchange two equations
          \item Add a multiple of one equation to another
        \end{enumerate}
  \item These operations are called \textbf{elementary row operations}
  \item Matrices that applied elementary row operation are still equivalence or row equivalence
\end{itemize}

\subsection*{Row-Echelon Form}
\begin{itemize}
  \item An augmented matrix in the form of Row-Echelon and Reduced Row-Echelon makes it easier to find the solution
  \item Reduced Row-Echelon characteristics:
        \begin{enumerate}
          \item The first element nonzero in a row is 1, called \textbf{leading 1}
          \item The next leading 1 in the lower row occurs farther to the right
          \item If any rows consist entirely of zeros, then they are grouped at the bottom of the matrix
          \item Each column that contains a leading 1 has zeros everywhere else in that column
        \end{enumerate}
  \item If the matrix only fulfilled 1, 2, and 3, then it is called row-echelon
\end{itemize}

\subsection*{Gauss–Jordan Elimination}
Gauss–Jordan Elimination method: convert the \underline{augmented matrix} into its \underline{equivalence} using elementary row operation in the \underline{reduced row-echelon form}.

\begin{enumerate}
  \item Let a linear system \(A=\)
        \begin{center}
          \(a_{11}x_1+a_{12}x_2+a_{13}x_3+\dots+a_{1n}x_n=b_1\) \\
          \(a_{21}x_1+a_{22}x_2+a_{23}x_3+\dots+a_{2n}x_n=b_2\) \\
          \(\vdots\) \\
          \(a_{m1}x_1+a_{m2}x_2+a_{m3}x_3+\dots+a_{mn}x_n=b_m\)
        \end{center}
  \item Convert using Gauss–Jordan Elimination with Elementary Row Operation
        \begin{center}
          \(\left[\begin{array}{cccc|c}
              a_{11} & a_{12} & \dots  & a_{1n} & b_1    \\
              a_{21} & a_{22} & \dots  & a_{2n} & b_2    \\
              \vdots & \vdots & \vdots & \vdots & \vdots \\
              a_{m1} & a_{m2} & \dots  & a_{mn} & b_m    \\
            \end{array}\right] \xRightarrow{\text{ERO}} \left[\begin{array}{cccc|c}
              1      & 0      & \dots  & 0      & b_1'   \\
              0      & 1      & \dots  & 0      & b_2'   \\
              \vdots & \vdots & \vdots & \vdots & \vdots \\
              0      & 0      & \dots  & 1      & b_m'   \\
            \end{array}\right] \)
        \end{center}
  \item The solutions are in reduced-row echelon form
\end{enumerate}
Note :
\begin{itemize}
  \item If a row is entirely zero except in the far right, then that system is inconsistent
        \begin{center}
          \(\left[\begin{array}{cccc|c}
              0      & 0      & \dots  & 0      & 2      \\
              0      & 1      & \dots  & 0      & b_2'   \\
              \vdots & \vdots & \vdots & \vdots & \vdots \\
              0      & 0      & \dots  & 1      & b_m'   \\
            \end{array}\right] \)
        \end{center}
  \item If the number of unknown is more than the number of nonzero rows, then that system has infinitely many solutions
        \begin{itemize}
          \item If a linear system has infinitely many solutions, then it has free parameter
        \end{itemize}
\end{itemize}

\section*{Homogeneous Linear Systems}
\begin{itemize}
  \item A linear system is said to be homogeneous if the constant terms are all zero, or in the other word, the numbers in the right of the equal sign are all zero
        \begin{center}
          \(\left[\begin{array}{cccc|c}
              a_{11} & a_{12} & \dots  & a_{1n} & 0      \\
              a_{21} & a_{22} & \dots  & a_{2n} & 0      \\
              \vdots & \vdots & \vdots & \vdots & \vdots \\
              a_{m1} & a_{m2} & \dots  & a_{mn} & 0      \\
            \end{array}\right]\)
        \end{center}
  \item Homogeneous Linear Systems is definitely consistent
  \item If the solution is \((x_1, x_2, x_3,\dots x_n)=(0,0,0,\dots,0)\), then this solution is called \textbf{trivial solution}, other than this is called \textbf{nontrivial solution}.
  \item A nontrivial solution occurred when there is at least one free parameter
\end{itemize}

\subsection*{Under-Determined and Over-Determined Linear System}
\begin{itemize}
  \item Under-determined linear system: Number of unknowns \(>\) numbers of the equation
        \begin{itemize}
          \item If it has a free parameter, then it has infinitely many solutions OR inconsistent
          \item Impossible to have a unique solution
        \end{itemize}
  \item Over-determined system of the linear equation: Number of unknown \(<\) number of equations
        \begin{itemize}
          \item Can has a unique solution, infinitely many solutions, or inconsistent
        \end{itemize}
\end{itemize}

\end{document}