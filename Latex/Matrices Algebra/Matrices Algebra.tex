\documentclass[a4paper,12pt]{article}
\usepackage[utf8]{inputenc} %codification of the document
\usepackage{mathtools}

\begin{document}

\title{Matrix Algebra}
% \maketitle

\section*{Matrix}
\begin{itemize}
  \item \textbf{Matrix} is a rectangle array of numbers consisting of \textbf{rows} and \textbf{columns}
  \item Matrix \(A\) can be written as \(a_{ij}\), where \(i\) is row and \(j\) are column, written as \(i\times j\)
  \item A matrix \(A\) with the order \(n \times n\), or a square matrix, has the entries consisting \(a_{11},a_{22},...,a_{nn}\) called \textbf{main diagonal} of \(A\)
\end{itemize}

\subsection*{Type of Matrices}
\begin{itemize}
  \item Square Matrix: Matrix with the same number of rows and the number of columns
  \item Zero Matrix: Matrix all of whose entries are zero
  \item Identity Matrix: Square matrix where its main diagonal is 1 and other elements are 0
  \item Column Matrix:
  \item Row Matrix:
\end{itemize}

\section*{Operations on Matrices}

\subsection*{Equality of Matrices}
Two matrices are defined to be equal if they have the same size and their corresponding entries are equal.

\subsection*{Addition and Subtraction}
Let \(A=(a_{ij})\) and \(B=(b_{ij})\) and has the same order \(m\times n\). Addition and subtraction of \(A\pm B\) are defined as :
\begin{center}
  \((A\pm B)_{ij}=A_{ij}\pm B_{ij}=a_{ij}\pm b_{ij}\)
\end{center}
The result elements are :
\begin{center}
  \(A+B=[a_{ij}+b_{ij}]\)
\end{center}
Note:
\begin{itemize}
  \item Addition and subtraction matrices could be done if and only if the order or size of said matrices is the same
\end{itemize}

\subsection*{Multiplication}
\subsubsection*{Scalar}
Let \(A=[a_{ij}]\) and scalar \(k\). Scalar multiplication \(kA\) defined as:
\begin{center}
  \((kA)_{ij}=k.(A)_{ij}=ka_{ij}\)
\end{center}
\subsubsection*{Two Matrices}
Let \(A=(a)_{ij}\) with the order \(m\times n\) and \(B=(b)_{ij}\) with the order \(p\times q\). If \(n=p\), then the product of AB is a matrix with the order \(m\times q\) and defined as :
\begin{center}
  \([AB]_{ij}=\sum\limits_{s=1}^{n} a_{is}b_{sj} \)
\end{center}
Note:
\begin{itemize}
  \item If \(n\neq p\) then product of AB is undefined
  \item The Cancellation Law does not hold
\end{itemize}


\subsection*{Exponential}
Let \(A=[a_{ij}]\) a square matrix of order \(n\), then for \(k \geq0\), defined as \(A^k=I\) for \(k=0\), and
\begin{center}
  \(A^k=\underbrace{A\times A\times A\times ... \times A}_{\text{k times}}\) \\
  \(A^{m+n}=A^nA^m\)
\end{center}

\section*{Properties of Matrices Algebra}
\begin{itemize}
  \item Addition matrices are commutative
        \begin{center}
          \(A+B=B+A\)
        \end{center}
  \item Addition matrices are associative
        \begin{center}
          \((A+B)+C=A+(B+C)\)
        \end{center}
  \item Addition matrices with zero matrix equal the matrix itself
        \begin{center}
          \(A+0=A\)
        \end{center}
  \item Each matrix \(A\) has negative \(-A\),with \(-A=-1(A)\) and \(A+(-A)=0\)
  \item Multiplication matrices with scalar are associative
        \begin{center}
          \(s(tA)=st(A)\)
        \end{center}
  \item Addition and Multiplication matrices with scalar are distributive
        \begin{center}
          \((A+B)C=AC+BC\) \\ \(r(A+B)=rA+rB\)
        \end{center}
\end{itemize}

\section*{Transpose}
Let a matrix \(A\) with the order \(m\times n\), then transpose of matrix \(A\), denoted by \(A^T\), is defined to be matrix with the order \(n\times m\) that results from interchanging the rows and columns of \(A\).
\subsection*{Properties}
\begin{itemize}
  \item Transpose of transpose of matrix A is equal to matrix A
        \begin{center}
          \((A^T)^T=A\)
        \end{center}
  \item Transpose of matrix are distributive
        \begin{center}
          \((A+B)^T=A^T+B^T\)
        \end{center}
  \item For scalar k, then
        \begin{center}
          \((kA)^T=k(A)^T\)
        \end{center}
  \item For two matrices A and B, then
        \begin{center}
          \((AB)^T=B^TA^T\)
        \end{center}
\end{itemize}
\subsection*{Symmetric Matrix}
If a matrix is a square matrix and that matrix is equal with its transpose, then it’s called a symmetric matrix.
\begin{center}
  \(A=A^T\)
\end{center}

\section*{Inverse Matrix}
\begin{itemize}
  \item A square matrix \(A\) is called has an inverse if there is \(B\) such that \begin{center}\( AB=BA=I\)\end{center}
  \item Matrix inverse is denoted by \(A^{-1}\)
  \item In the case above, the inverse of A is B, or \(A^{-1}=B\)
\end{itemize}

\subsection*{The Inverse of \(2\times 2\) Matrix}
Let \(A=\begin{bmatrix}a & b \\ c & d\end{bmatrix}\), then \(A^{-1}=\frac{1}{ab-cd}\begin{bmatrix}d & -b \\ -c & a\end{bmatrix}\)

\subsection*{Orthogonal Matrix}
\begin{itemize}
  \item Matrix \(A\) is orthogonal if and only if \(A^{T}=A^{-1}\)
  \item \((A^{-1})^T=(A^T)^{-1}\)
\end{itemize}

\subsection*{Properties of Matrix Inverse}
\begin{itemize}
  \item The inverse of a matrix is \textbf{unique}, or there is \textbf{only one} inverse for each matrix
  \item Inverse of inverse matrix is equal the matrix itself\begin{center}\((A^{-1})^{-1}=A\)\end{center}
  \item If \(A\) has an inverse and \(n>0\), then \(A^n\) has an inverse and \((A^n)^{-1}=(A^{-1})^n\)
  \item For \(k\neq 0\), then \((kA)^{-1}=\frac{1}{k}A^{-1}\)
  \item Matrix \(B_{m\times n}\) has an inverse, then \(AB\) has an inverse and \((AB)^{-1}=B^{-1}A^{-1}\)
\end{itemize}

\section*{Elementary Matrix}
An elementary matrix is a matrix that differs from the identity matrix by \textbf{exactly one} elementary row operation. For example: \\ \\
\(I=\begin{bmatrix}
  1 & 0 & 0 & 0 \\ 0 & 1 & 0 & 0 \\ 0 & 0 & 1 & 0 \\ 0 & 0 & 0 & 1
\end{bmatrix}\) $\xRightarrow{R_2 \leftarrow 7\times R_2}$ \(\begin{bmatrix}
  1 & 0 & 0 & 0 \\ 0 & 7 & 0 & 0 \\ 0 & 0 & 1 & 0 \\ 0 & 0 & 0 & 1
\end{bmatrix}\)

\subsection*{Inverse of Elementary Matrix}
\begin{itemize}
  \item Consider this matrix.
        \begin{itemize}
          \item \(E=\begin{bmatrix}
                  1 & 0 & 0 \\ 0 & 3 & 0 \\ 0 & 0 & 1
                \end{bmatrix}\) is a elementary matrix from \(I\) by applying a single row operation \(R_2 \leftarrow 3R_2\)
          \item Inverse of matrix E, denoted by \(E^{-1}\), can be obtained from \(I\) by applying the inverse of said single row operation, \(R_2 \leftarrow \frac{1}{3}R_2\)
        \end{itemize}
  \item So, the inverse elementary matrix can be obtained by applying the inverse single row operation
  \item Each elementary matrix has an inverse and its inverse is also an elementary matrix
\end{itemize}

\subsection*{Multiplication with Elementary Matrix}
Let square matrices \(E\) and \(A\) with the same order \(m\times m\), with \(E\) is an elementary matrix by applying a single row operation, named \(R\). The product of \(EA\) is equal with applying single elementary row operation \(R\) to \(A\).

\section*{Finding Inverse with Elementary Row Operation}
If matrix \(A\) has an inverse, then \(A\) can be reduced to row echelon form by performing a sequence of elementary row operation to obtained the inverse of \(A\), or \(A^{-1}\) \\
For example: Let \(A=\begin{bmatrix}
  4 & 2 \\ 2 & 2
\end{bmatrix}\) \\ \\
\begin{enumerate}
  \item Reduce matrix \(A\) to row echelon form \((A | I)\)
  \item Perform a sequence of elementary row operation:
        \begin{enumerate}
          \item \(\left[\begin{array}{cc|cc}
                    4 & 2 & 1 & 0 \\ 2 & 2 & 0 & 1
                  \end{array}\right] \) $\xRightarrow{R_1 \leftarrow \frac{1}{4}R_1}$\(\left[\begin{array}{cc|cc}
                    1 & \frac{1}{2} & \frac{1}{4} & 0 \\ 2 & 2 & 0 & 1
                  \end{array}\right] \)
          \item \(\left[\begin{array}{cc|cc}
                    1 & \frac{1}{2} & \frac{1}{4} & 0 \\ 2 & 2 & 0 & 1
                  \end{array}\right] \) $\xRightarrow{R_2 \leftarrow R_2 - 2R_1}$ \(\left[\begin{array}{cc|cc}
                    1 & \frac{1}{2} & \frac{1}{4} & 0 \\ 0 & 1 & -\frac{1}{2} & 1
                  \end{array}\right] \)
          \item \(\left[\begin{array}{cc|cc}
                    1 & \frac{1}{2} & \frac{1}{4} & 0 \\ 0 & 1 & -\frac{1}{2} & 1
                  \end{array}\right] \) $\xRightarrow{R_1 \leftarrow R_1-\frac{1}{2}R_2}$ \(\left[\begin{array}{cc|cc}
                    1 & 0 & \frac{1}{2} & -\frac{1}{2} \\ 0 & 1 & -\frac{1}{2} & 1
                  \end{array}\right] \)
        \end{enumerate}
  \item The result is \((I|A^{-1})=\left[\begin{array}{cc|cc}
            1 & 0 & \frac{1}{2} & -\frac{1}{2} \\ 0 & 1 & -\frac{1}{2} & 1
          \end{array}\right]\), or \(A^{-1}=\begin{bmatrix}
          \frac{1}{2} & -\frac{1}{2} \\ -\frac{1}{2} & 1
        \end{bmatrix}\)
\end{enumerate}

\section*{Solving System of Linear Equation with Inverse Matrix}
If matrix \(A\) has an inverse, then for each matrix \(B\) with the order \(n\times 1\), system of linear equation \(Ax=B\) has unique solution, with \(x=A^{-1}B\)

\end{document}