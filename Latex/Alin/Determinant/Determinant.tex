\documentclass[a4paper,12pt]{article}
\usepackage[utf8]{inputenc} %codification of the document
\usepackage{mathtools}
\usepackage{amsfonts}
\usepackage{amsmath}
\usepackage{amssymb}

\begin{document}

\title{Determinant}
% \maketitle

\section*{Determinant}
\begin{itemize}
  \item Determinant: a \textbf{scalar value} that is a function of the entries of a \textbf{square matrix}
  \item Let A be a matrix, then its determinant is denoted by \(det(A)\), \(det A\), or \(|A|\)
  \item The importance of the determinant
        \begin{itemize}
          \item Indicates that a matrix has an inverse or not
          \item Indicates that a linear system has a unique solution or not
          \item Plays an important role in determining the values and eigenvectors
          \item etc.
        \end{itemize}
\end{itemize}

\section*{Calculate The Determinant of A Matrix}

\subsection*{Rule of Sarrus}
\begin{itemize}
  \item \(2\times 2\) Matrix
  \item[] Let \(A=\begin{bmatrix}
      a_{11} & a_{12} \\ a_{21} & a_{22}
    \end{bmatrix}\), then \(det(A)=(a_{11}.a_{22})-(a_{12}.a_{21})\)
  \item \(3\times 3\) Matrix
  \item Let \(A=\begin{bmatrix}
          a_{11} & a_{12} & a_{13} \\ a_{21} & a_{22} & a_{23} \\ a_{31} & a_{32} & a_{33}
        \end{bmatrix}\), then:
        \begin{center}
          \(det(A)=((a_{11}.a_{22}.a_{23})+(a_{12}.a_{23}.a_{31})+(a_{13}.a_{21}.a_{32}))-((a_{13}.a_{22}.a_{31})+(a_{11}.a_{23}.a_{32})+(a_{12}.a_{21}.a_{33}))\)
        \end{center}
\end{itemize}
Note:
\begin{itemize}
  \item Advantage: Simple for \(2\times 2\) and \(3\times 3\) matrices
  \item Disadvantage: For a larger matrix, it would be very troublesome to calculate the determinant
\end{itemize}

\subsection*{Cofactor Expansion}
\subsubsection*{Minor and Cofactor}
\begin{itemize}
  \item Minor \(M_{ij}\) is the determinant of matrix A after removing the \(i\)-th row and \(j\)-th column
  \item The cofactor \(C_{ij}\) is \((-1)^{i+j}M_{ij}\)
  \item For example, let \(A=\begin{bmatrix}
          a_{11} & a_{12} & a_{13} \\ a_{21} & a_{22} & a_{23} \\ a_{31} & a_{32} & a_{33}
        \end{bmatrix}\)
        \begin{itemize}
          \item[-] \(M_{13}=det\begin{bmatrix}
              a_{21} & a_{22} \\ a_{31} & a_{32}
            \end{bmatrix}\)
          \item[] \(C_{13}=(-1)^{1+3}M_{13}\)
          \item[-] \(M_{21}=det\begin{bmatrix}
              a_{12} & a_{13} \\ a_{32} & a_{33}
            \end{bmatrix}\)
          \item[] \(C_{21}=(-1)^{2+1}M_{21}\)
        \end{itemize}
\end{itemize}
\subsubsection*{Row and Column Expansion}
\begin{itemize}
  \item Based on the formula derivation from the Rule of Sarrus, there is a pattern
  \item This derived the cofactor expansion : \begin{center}
          \(det(A) = \sum\limits_{j=1}^{n} a_{ij}C_{ij} = \sum\limits_{i=1}^{n} a_{ij}C_{ij}\)
        \end{center}
  \item[] For example, let \(A=\begin{bmatrix}
      a_{11} & a_{12} & a_{13} \\ a_{21} & a_{22} & a_{23} \\ a_{31} & a_{32} & a_{33}
    \end{bmatrix}\), then:
    \begin{itemize}
      \item Row 1 Expansion: \(det(A) = a_{11}C_{11}+a_{12}C_{12}+a_{13}C_{13}\)
      \item Column 3 Expansion: \(det(A) = a_{13}C_{13}+a_{23}C_{23}+a_{33}C_{33}\)
    \end{itemize}
\end{itemize}
Note: Sometimes, it is important to examine the matrix first to find the easiest row or column to calculate the determinant

\subsection*{Combinatorics}
TBA

\subsection*{Elementary Row Operation}
\subsubsection*{Effect of Elementary Row Operations on Determinants}
If \(X'\) obtained from matrix \(X\) by applying an elementary row operation \(R\), then: \begin{center}
  \begin{tabular}{|c|l|} \hline
    \textbf{Elementary Row Operation}         & \textbf{Effect on Determinant} \\ \hline
    \(R_i\leftrightarrow R_j\)                & \(det(X')=-1.det(X)\)          \\ \hline
    \(R_i\leftarrow k.R_i, k\neq 0 \)         & \(det(X')=k.det(X)\)           \\ \hline
    \(R_i\leftarrow k.R_i+l.R_j, k,l\neq 0 \) & \(det(X')=det(X)\)             \\ \hline
  \end{tabular}
\end{center}
\subsubsection*{Elementary Row Operation}
Let \(A\) a matrix, \(I\) is reduced row-echelon form of A, \(r\) is the times interchange row operations, \(s\) is the times multiply the equation with nonzero constant \(k_1,k_2,\dots\ k_s\), and \(t\) is the adding a multiple of one equation to another. The determinant of A can be calculate from the elementary row operations that used to change the matrix from A to I as follows:
\begin{center}
  \(det(A)=\frac{(-1)^r}{(k_1k_2\dots k_s)}\)
\end{center}

\subsection*{Simple Matrices}
There are 'cheat' for certain matrices.
\begin{itemize}
  \item Diagonal Matrix
  \item[] \(A=\begin{bmatrix}
      9 & 0 & 0 \\ 0 & 7 & 0 \\ 0 & 0 & 8
    \end{bmatrix}\), \(det(A)=9\times 7\times 8=504\)
  \item Upper Triangular Matrix
  \item[] \(B=\begin{bmatrix}
      1 & 2 & 3 \\ 0 & 2 & -4 \\ 0 & 0 & 5
    \end{bmatrix}\), \(det(A)=1\times 2\times 5=10\)
  \item Matrix with Row or Column Zero
  \item[] \(C=\begin{bmatrix}
      1 & 6 & 7 \\ 2 & 4 & 8\\ 0 & 0 & 0
    \end{bmatrix}\), \(det(A)=0\)
  \item[] \(D=\begin{bmatrix}
      1 & 0 & 7 \\ 2 & 0 & 8\\ 3 & 0 & 9
    \end{bmatrix}\), \(det(A)=0\)
  \item Matrix with Identical Row
  \item[] \(E=\begin{bmatrix}
      1 & 4 & 7 \\ 1 & 4 &7\\ 3 & 8 & 9
    \end{bmatrix}\), \(det(A)=0\)
\end{itemize}

\section*{Properties of Determinant}
\begin{itemize}
  \item \(det(AB)=det(A).det(B)\)
  \item \(det(A+B)\neq det(A)+det(B)\)
  \item \(det(A^T)=det(A)\)
  \item \(det(A)=\frac{1}{det(A^{-1})}\)
  \item Let \(A\) a square matrix of order \(n\), then \(det(kA)=k^n det(A)\)
\end{itemize}

\section*{Cramer's Rule}

\subsection*{Adjoint Matrix}
\begin{itemize}
  \item Let \(A\) be a square matrix of order \(n\). The adjoint of matrix \(A\) is the \underline{transpose} of the \underline{cofactor matrix} of \(\).
  \item Denoted by \(adj A\)
  \item Also called Adjugate Matrix
\end{itemize}
\begin{center}
  \(A=\begin{bmatrix}
    a_{11} & a_{12} & \dots  & a_{1n} \\
    a_{21} & a_{22} & \dots  & a_{2n} \\
    \vdots & \vdots & \ddots & \vdots \\
    a_{n1} & a_{n2} & \dots  & a_{nn} \\
  \end{bmatrix} \Rightarrow [C_{ij}]=\begin{bmatrix}
    C_{11} & C_{12} & \dots  & C_{1n} \\
    C_{21} & C_{22} & \dots  & C_{2n} \\
    \vdots & \vdots & \ddots & \vdots \\
    C_{n1} & C_{n2} & \dots  & C_{nn} \\
  \end{bmatrix} \Rightarrow \) \\
  \(adj A=[C_{ij}]^T=\begin{bmatrix}
    C_{11} & C_{21} & \dots  & C_{n1} \\
    C_{12} & C_{22} & \dots  & C_{n2} \\
    \vdots & \vdots & \ddots & \vdots \\
    C_{1n} & C_{2n} & \dots  & C_{nn} \\
  \end{bmatrix}\)
\end{center}

\subsubsection*{Cramer’s Rule}
\begin{itemize}
  \item Cramer’s Rule: A method that uses determinants to solve systems of equations that have the same number of equations as variables.
  \item Consider a linear system \(Ax=b\) and \(A\) has an inverse. Then:
  \begin{align*}
    x & = A^{-1} b \\
    & = \Bigg(\frac{1}{det(A)}.adj A\Bigg)b \\ 
    \begin{bmatrix}
      x_1 \\ x_2 \\ \vdots \\ x_n
    \end{bmatrix} & = \frac{1}{det(A)}.\begin{bmatrix}
      C_{11} & C_{21} & \dots  & C_{n1} \\
      C_{12} & C_{22} & \dots  & C_{n2} \\
      \vdots & \vdots & \ddots & \vdots \\
      C_{1n} & C_{2n} & \dots  & C_{nn} \\
    \end{bmatrix} \begin{bmatrix}
      b_1 \\ b_2 \\ \vdots \\ b_n
    \end{bmatrix} \\
    & = \frac{1}{det(A)}.\begin{bmatrix}
      b_1C_{11} & b_2C_{21} & \dots  & b_nC_{n1} \\
      b_1C_{12} & b_2C_{22} & \dots  & b_nC_{n2} \\
      \vdots & \vdots & \ddots & \vdots \\
      b_1C_{1n} & b_2C_{2n} & \dots  & b_nC_{nn} \\
    \end{bmatrix} \\ 
    & = \frac{1}{det(A)}.\begin{bmatrix}
      det(A_1) \\ det(A_2) \\ \vdots \\ det(A_n)
    \end{bmatrix}
  \end{align*}
  \item[] \(\therefore x_j=\frac{det(A_j)}{det(A)}\), for \(j=1,2,\dots n\)
\end{itemize}
Note: Because determinant of coefficient matrix are used as divisor, then Cramer's Rule can be applied if \textbf{the coefficient matrix is square matrix} and its \textbf{determinant is nonzero} (or \textbf{the coefficient matrix has inverse}).

\end{document}